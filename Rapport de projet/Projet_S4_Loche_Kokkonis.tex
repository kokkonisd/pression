\documentclass{polytech/polytech}

%-----------Configuration du rapport-----------------

\schooldepartment{peip}
\typereport{peip2}
\reportyear{2016-2017}
\title{DI 6 Correcteur de posture assise à base d'Arduino}
\reportlogo{image/logo_1ere_couverture} %A changer avec la photo de 1ere de couverture
\student{Jérémy}{LOCHE}{jeremy.loche@etu.univ-tours.fr}
\student{Dimitrios}{KOKKONIS}{dimitrios.kokkonis@etu.univ-tours.fr}
\academicsupervisor[di]{Sébastien}{BEAUFILS}{sebastien.beaufils@univ-tours.fr}

%------------Poster------------------

\posterblock{Titre 1 poster}{texte 1 poster}{image/poster_1}{légende 1 poster}

\posterblock{Titre 2 poster}{texte 2 poster}{image/poster_1}{légende 2 poster}

\posterblock{Titre 3 poster}{texte 3 poster}{image/poster_1}{légende 3 poster}




%------------------Résumés & mots clés------------

\resume{Ce projet a pour but de créer un correcteur de posture assise d'une personne, sous la forme d'un système embarqué. Le projet est basé sur un Arduino Uno, qui gère les données de quatre capteurs de charges placées sous les pieds d'une chaise. On l'utilise ensuite pour transmettre ces données en série (soit par Bluetooth, soit par USB) à une application fonctionnant sur un ordinateur Windows/OSX/Linux ou un appareil Android. Cette application permet de visualiser la chaise, la position du barycentre de l'utilisateur dans une zone de confort calibrée par l'utilisateur.  On peut ainsi détecter si la posture de l'utilisateur est bonne; l'application l'informe visuellement quand ce n'est pas le cas. Le logiciel utilisé est écrit en Java et en C.}



\abstract{This project aims to create an application for correcting one's sitting posture, using both software and hardware. The project was built upon an Arduino Uno that handles the data sent by four load sensors placed under the feet of a chair, and which in turn modifies and sends that data in serial form (either by Bluetooth or by USB) to an application that runs on a Windows/OSX/Linux computer or an Android device. That device then provides the user with a graphic representation of the chair, the barycenter of the user, as well as a "deadzone" which is calibrated by the user, and is used to detect wether the user's posture is incorrect or not; the application informs the user visually when he is not properly seated. The software is written in Java and C.}


\motcle{Correction de posture, Système embarqué, Arduino, Java, C}

\keyword{Posture correction, Embedded system, Arduino, Java, C}


\begin{document}

\chapter*{Introduction}

De nos jours, de nombreuses personnes souffrent du mal de dos dut à une mauvaise posture assise. En effet, il suffit de se rendre dans une salle de cours pleine d'étudiants pour constater que beaucoup d'entre eux sont mal assis. C'est pourquoi, nous avons choisi de travailler sur le projet du développement d'un correcteur de posture assise proposé par M.Beaufils. La proposition de M. Beaufils a été de construire ce dispositif à partir d'un Arduino Uno et des capteurs de forces placés sous les pieds d'une chaise. 

Au cours de la lecture de ce rapport, vous pourrez en apprendre plus sur notre proposition de réalisation de ce projet. En effet, vous découvrirez comment nous avons choisis de répondre à ce problème.

\chapter{Le matériel et Arduino}

\chapter{Application}

%% === COMPTES RENDUS ===
\weeklyreport{11/01/2017}{Rapport seance 1}
\weeklyreport{18/01/2017}{Rapport seance 2}
\weeklyreport{25/01/2017}{Rapport seance 3}
\weeklyreport{01/02/2017}{Rapport seance 4}
\weeklyreport{08/02/2017}{Rapport seance 5}
\weeklyreport{15/02/2017}{Rapport seance 6}
\weeklyreport{08/03/2017}{Rapport seance 7}
\weeklyreport{15/03/2017}{Rapport seance 8}
\weeklyreport{22/03/2017}{Rapport seance 9}
\weeklyreport{29/03/2017}{Rapport seance 10}
\weeklyreport{05/04/2017}{Rapport seance 11}
\weeklyreport{26/04/2017}{Rapport seance 12}

\end{document}


