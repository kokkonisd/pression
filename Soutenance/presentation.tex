% how-to guide: https://www.sharelatex.com/blog/2013/08/13/beamer-series-pt1.html

\documentclass{beamer}
\usetheme{Warsaw}
\usepackage[utf8]{inputenc}
\usepackage[french]{babel}
\usepackage{graphicx}

\title{Correcteur de posture assise à base d'Arduino}
\subtitle{\small DI 6}
\author[KOKKONIS Dimitrios \\\and LOCHE Jérémy]{\small {\small \texttt{Auteurs}}\\ KOKKONIS Dimitrios \\\and LOCHE Jérémy\\
\vspace{5px}
{\small \texttt{Encadrant}}\\ BEAUFILS Sébastien}
\institute{\textsc{École Polytechnique de l'Université de Tours}}
\date{11 mai 2017}

% new environments
\newenvironment{figim}[2]{%
	\begin{figure}[htbp]
	\caption{#1}
	\label{#2}
	\begin{center}
}{%
	\end{center}
	\end{figure}
}

\begin{document}

\begin{frame}
\vspace{0.1cm}
\includegraphics[height=1cm]{images/Logo_PeiP_v2009_RGB_3cm_300dpi.jpg}
\includegraphics[height=1cm]{images/logo_Polytech_Tours_RVB_3cm_300dpi.jpg}
\hfill
\includegraphics[height=1cm]{images/logo_UFR_4cm_300dpi.jpg}
\titlepage
\end{frame}

\begin{frame}
\frametitle{Plan}
\tableofcontents
\end{frame}

\section*{Introduction}
\begin{frame}
\begin{block}{Introduction}
\begin{itemize}
\pause
\item Objectif :
\pause
\begin{itemize}
\item Créer un correcteur de posture assise
\pause
\item Communiquer un résultat
\end{itemize}
\pause
\item Solution apportée :
\begin{itemize}
\pause
\item Création d'un système embarqué 
\end{itemize}
\end{itemize}
\end{block}
\end{frame}

\section{La communication en série}
\begin{frame}
\begin{block}{La communication en série}
\pause
blablabla
\end{block}
\end{frame}

\section{Les cellules de charge}
\begin{frame}
\begin{block}{Les cellules de charge}
\pause
blablabla
\end{block}
\end{frame}

\section{L'Arduino}
\begin{frame}
\begin{block}{L'Arduino}
\pause
blablabla
\end{block}
\end{frame}

\section{La modélsation et la réalisation du correcteur de posture}
\begin{frame}
\begin{block}{La modélsation et la réalisation du correcteur de posture}
\pause
blablabla
\end{block}
\end{frame}

\section{Conclusion}
\begin{frame}
\begin{block}{Conclusion}
\pause
blablabla
\end{block}
\end{frame}

\subsection*{Bibliographie}
\begin{frame}
\frametitle{Bibliographie}
\begin{thebibliography}{99}
{\tiny \bibitem{site-arduino} Arduino, \textit{Le site officiel d'Arduino}. 2017. \texttt{\tiny URL : https://www.arduino.cc/} (visité le 26/04/2017).

\bibitem{ntu-swing} Nanyang Technological University, \textit{Java Programming Tutorial - Custom Graphics}. 2016. \texttt{\tiny URL : http://www.ntu.edu.sg/home/ehchua/programming/java/J4b\_CustomGraphics.html} (visité le 26/04/2017).

\bibitem{load-cell} Electronics Stack Exchange, \textit{How to mount a Half Bridge Load cell}. 2016. \texttt{\tiny URL : https://electronics.stackexchange.com/questions/233574/how-to-mount-a-half-bridge-load-cell} (visité le 26/04/2017).

\bibitem{hx711} bodge, \textit{HX711 Arduino library}. 2016. \texttt{\tiny URL : https://github.com/bogde/HX711} (visité le 26/04/2017).

\bibitem{android-studio} Google, \textit{Android Studio}. 2017. \texttt{\tiny URL : https://developer.android.com/studio/index.html} (visité le 26/04/2017).

\bibitem{repo-github} KOKKONIS Dimitrios, LOCHE Jérémy, \textit{pression}. 2017. \texttt{\tiny URL : https://github.com/kokkonisd/pression} (visité le 26/04/2017).

\bibitem{sparkfun} Sparkfun, \textit{Getting Started with Load Cells}. 2016. \texttt{\tiny URL : https://learn.sparkfun.com/tutorials/getting-started-with-load-cells} (visité le 26/04/2017).

\bibitem{wheatstone} National Instruments, \textit{Comment effectuer une mesure avec une cellule de charge ou un transducteur de pression}. 2013. \texttt{\tiny URL : http://www.ni.com/tutorial/7138/fr/} (visité le 26/04/2017).}
\end{thebibliography}
\end{frame}

\subsection*{Liens utiles}
\begin{frame}
\frametitle{Liens utiles}
\begin{itemize}
\item \url{www.polytech.univ-tours.fr}
\end{itemize}
\end{frame}

\end{document}
