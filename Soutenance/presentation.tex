% how-to guide: https://www.sharelatex.com/blog/2013/08/13/beamer-series-pt1.html

\documentclass{beamer}
\usetheme{Warsaw}
\usepackage[utf8]{inputenc}
\usepackage[french]{babel}
\usepackage{graphicx}
\usepackage{tikz}
\usepackage{eurosym}

\definecolor{white}{rgb}{255,255,255}

\setbeamerfont{page number in head/foot}{size=\small}

\setbeamertemplate{footline}[frame number]

\title{Correcteur de posture assise à base d'Arduino}
\subtitle{\small DI 6}
\author[KOKKONIS Dimitrios \\\and LOCHE Jérémy]{\small {\small \texttt{Auteurs}}\\ KOKKONIS Dimitrios \\\and LOCHE Jérémy\\
\vspace{5px}
{\small \texttt{Encadrant}}\\ BEAUFILS Sébastien}
\institute{\textsc{École Polytechnique de l'Université de Tours}}
\date{11 mai 2017}

% new environments
\newenvironment{figim}[2]{%
	\begin{figure}[htbp]
	\caption{#1}
	\label{#2}
	\begin{center}
}{%
	\end{center}
	\end{figure}
}

\begin{document}

\begin{frame}
\vspace{0.1cm}
\includegraphics[height=1cm]{images/Logo_PeiP_v2009_RGB_3cm_300dpi.jpg}
\includegraphics[height=1cm]{images/logo_Polytech_Tours_RVB_3cm_300dpi.jpg}
\hfill
\includegraphics[height=1cm]{images/logo_UFR_4cm_300dpi.jpg}
\titlepage
\end{frame}



\section*{Introduction}
\begin{frame}
\begin{block}{Introduction}
Problématique:

\begin{center}
\textit{Concevoir un correcteur de posture assise à l'aide d'un Arduino. }
\end{center}

\begin{itemize}
\item Objectifs :
\begin{itemize}
\item Déterminer la posture assise d'une personne à l'aide de capteurs placés sous les pieds de la chaise
\item Créer une interface utilisateur adaptée pour communiquer la qualité de la posture
\end{itemize}
\item Solution apportée :
\begin{itemize}
\item Création d'un système embarqué 
\item Création d'un logiciel  PC et Android adapté à la visualisation de la posture
\end{itemize}
\end{itemize}
\end{block}
\end{frame}


\begin{frame}
\frametitle{Plan}
\tableofcontents
\end{frame}

\section{Déterminer la posture assise d'une personne ?}
\begin{frame}
\frametitle{\textcolor{white}{Déterminer la posture assise d'une personne ?}}
Pourquoi utiliser des capteurs sous les pieds de la chaise ?
\begin{figure}
\begin{center}
\includegraphics[height=5cm]{images/Chaise_forces_homo.jpg}
\includegraphics[height=5cm]{images/Chaise_forces_hetero.jpg}
\end{center}
\caption{Répartition des forces sur une chaise en fonction de la posture}
\label{fig:chaise_repartition}
\end{figure}
\end{frame}


\begin{frame}
\begin{block}{Les problèmes :}
\begin{enumerate}
\item Quels capteurs choisir pour mesurer la force sous les pieds de la chaise ?
\item Comment mesurer la valeur de ces capteurs ?
\item Comment faire savoir à l'utilisateur la qualité de sa posture assise ?
\end{enumerate}
\end{block}
\end{frame}


\begin{frame}
\begin{block}{La solution: le système embarqué et son logiciel}
\begin{itemize}
\item 4 cellules de charge %dont la résistance varie lorsqu'on applique une force sur eux
\item 4 ponts Wheatstone %(construits à partir des cellules de charge) qui permettent de mésurer une tension
\item 4 amplificateurs HX711 %qui amplifient le signal fourni par les cellules de charge
\item 1 Arduino Uno et son Shield Bluetooth%, qui récupère les données fournies par les amplificateurs, les traite, et les envoie à l'application
\item une application (PC ou Android), qui reçoit les données de l'Arduino (par Bluetooth ou par USB) et permet de visualiser la posture à partir de ces données
\end{itemize}
\end{block}
\end{frame}

\section{Le micro-contrôleur Arduino et ses capteurs}

\subsection{Les cellules de charge}
\begin{frame}
%\begin{block}{Les cellules de charge}
\frametitle{\textcolor{white}{Les cellules de charge}}
\begin{itemize}
\item Il s'agit de résistance variables qui varient en fonction de la déformation du métal.
\item $R=\rho \frac{l}{s}$
\item Les variations de la résistance sont toutefois très faibles ($m\Omega$ ou $\mu\Omega$)
\end{itemize}
\begin{figure}
\begin{center}
\includegraphics[height=2.8cm]{images/load_cell.jpg}
\end{center}
\caption{Schéma d'une cellule de charge avec jauge de contrainte (image : \cite{sparkfun}).}
\label{fig:load_cell_sparkfun}
\end{figure}
%\end{block}
\end{frame}

\begin{frame}
\begin{block}{Notre choix de cellule de charge}
4 cellules de charges récupérés dans un pèse personne électronique.
\begin{itemize}
\item Capacité de la cellule: jusqu'à 50 kg 
\item Capacité totale du correcteur de posture: jusqu'à 200 kg
\end{itemize}
\begin{figure}
\begin{center}
\includegraphics[height=3.5cm]{images/load_sensor.jpg}
\includegraphics[height=3.5cm]{images/load_sensor2.jpg}
\end{center}
\caption{Cellule de charge de pèse personne électronique}
\label{fig:load_cell_1}
\end{figure}
\end{block}
\end{frame}


\begin{frame}
\begin{block}{Comment mesurer les variations de résistances?}
\begin{itemize}
\item Le pont Wheatstone
\begin{itemize}
\item Permet de transformer la mesure de la résistance en une mesure de tension
\item Toutefois le signal est aussi très faible (de l'ordre du $mV$)
\end{itemize}
\end{itemize}

\begin{figure}
\begin{center}
\includegraphics[height=3cm]{images/wheatstone_bridge.jpg}
\end{center}
\caption{Schéma d'un pont de Wheatstone avec $V_{EX}$ la tension d'alimentation et $V_O$ la tension de sortie (site internet de source : \cite{wheatstone}).}
\label{fig:wheatstone_bridge}
\end{figure}
\end{block}
\end{frame}

\begin{frame}
\begin{block}{Comment amplifier le signal ?}
\begin{itemize}
\item Les amplificateurs HX711
\begin{itemize}
\item Prennent un pont de wheatstone en entrée
\item On peut les connecter aux pins digitales de l'Arduino
\end{itemize}
\end{itemize}

\begin{figure}
\begin{center}
\includegraphics[height=3cm]{images/HX711.jpg}
\includegraphics[height=3cm]{images/load_sensor_connected.jpg}
\end{center}
\caption{4 cellules de charges en demi-pont connectées à l'Arduino en utilisant des amplis HX711.}
\label{fig:load_sensor_connected}
\end{figure}
\end{block}
\end{frame}

\subsection{Le micro-contrôleur Arduino}
\begin{frame}
\frametitle{\textcolor{white}{Le micro-contrôleur Arduino}}
\begin{itemize}
\item Capable de lire des données analogiques ou numériques, les traiter, et les envoyer par série
\item Utilise un compilateur \texttt{avr-g++} (C/C++)
\item Peut utiliser un shield Bluetooth qui permet d'envoyer des données par Bluetooth
\end{itemize}
\begin{figure}
\begin{center}
\includegraphics[height=3cm]{images/uno.jpg}
\includegraphics[height=3cm]{images/BTshield.jpg}
\end{center}
\caption{Arduino Uno et son Shield bluetooth}
\label{fig:load_sensor_connected}
\end{figure}
\end{frame}

\begin{frame}
\begin{block}{La programmation de l'Arduino}
\begin{itemize}
\item Le code C/C++: écrit dans Arduino IDE
\item Récupère les valeurs des capteurs et les envoie via USB et Bluetooth toutes les 100ms.
\item Il utilise une libraire développée pour les amplificateurs, appelée \textit{HX711} \cite{hx711}
\item Le code est aussi capable de faire la tare des capteurs (faire le zéro)
\end{itemize}

\end{block}
\end{frame}

\section{Les applications}
\begin{frame}
\frametitle{\textcolor{white}{Les applications}}
\begin{itemize}
\item 2 applications différentes :
\begin{itemize}
\item Application PC (Windows/OSX/Linux)
\item Application Android
\item Construites en \texttt{Java} avec \textit{Eclipse} et \textit{Android Studio}
\item Développement partagé avec \textit{Github}
\end{itemize}

\begin{figure}
\includegraphics[height=2cm]{images/eclipse.png}
\hspace{1cm}
\includegraphics[height=2cm]{images/android-studio.png}
\hspace{1cm}
\includegraphics[height=2cm]{images/github.png}
\end{figure}

\item Se basent sur 2 fonctionnalités principales :
\begin{itemize}
\item La communication série $TX \rightarrow RX$
\item La visualisation de la posture (l'interface utilisateur)
\end{itemize}
\end{itemize}

\end{frame}

\subsection{La communication série}
\begin{frame}
\frametitle{\textcolor{white}{La communication en série}}

\begin{figure}
\includegraphics[height=5.5cm]{images/rxtx.png}
\caption{Principe de la communication série}
\end{figure}
\end{frame}

%\begin{frame}
%Application en JAVA:
%\begin{itemize}
%\item La librairie \texttt{jSerialComm}
%\item Elle permet de :
%\begin{itemize}
%\item Trouver les ports séries disponibles sur la machine
%\item Se connecter aux ports et établir des flux de communication (envoyer/recevoir des données)
%\item Interpréter les données d'entrée
%\item Fermer le flux de communication et se déconnecter
%\end{itemize}
%\end{itemize}
%\end{frame}

%\subsection{La modélisation de la chaise}
%\begin{frame}
%\begin{block}{La modélisation de la chaise}
%\begin{itemize}
%\item Objet "parent" : chaise
%\begin{itemize}
%\item Objet "enfant" : pied
%\begin{itemize}
%\item Comporte un ID unique associé à l'amplificateur correspondant
%\item Comporte une position
%\item Comporte une valeur
%\end{itemize}
%\item Comporte une "zone optimale" (appelée "deadzone") :
%\begin{itemize}
%\item Position
%\item Rayon
%\end{itemize}
%\item Comporte un barycentre
%\end{itemize}
%\end{itemize}
%\end{block}
%\end{frame}

%\begin{frame}
%\begin{block}{Comment changer de référentiel ?}
%\begin{figure}[htbp]
%\includegraphics[height=4cm]{images/refs}
%\caption{Le même point exprimé en deux référentiels : \textit{réel} (à gauche) et \textit{numérique} (à droite).}
%\label{fig:visualisation_referentiels}
%\end{figure}
%\end{block}
%\end{frame}

\subsection{La visualisation}
\begin{frame}
\frametitle{\textcolor{white}{La visualisation}}
\begin{figure}[htbp]
\begin{center}
\includegraphics[height=4.2cm]{images/screenshot_pc1}
\hspace{0.1cm}
\includegraphics[height=4.2cm]{images/screenshot_android1.png}
\end{center}
\caption{L'application PC (gauche) et Android (droite)}
\label{fig:screenshot_pc}
\end{figure}
\end{frame}

\section{Conclusion}

\subsection{Consommation énergétique}

\begin{frame}
\frametitle{\textcolor{white}{Consommation énergétique}}
\begin{table}
\begin{center}
\begin{tabular}{| c | c |}
\hline
\textbf{Matériel} & \textbf{Consommation} \\\hline\hline
Arduino & 50 mA \\\hline
Shield Bluetooth & 30 mA \\\hline
$4 \times$HX711 + Cellules de charge & 25 mA \\\hline
\textbf{Total} &  \textbf{105 mA} \\\hline
\end{tabular}
\end{center}
\caption{Consommation théorique de chaque partie du montage électronique}
\label{tab:consommation}
\end{table}


\begin{table}
\begin{center}
\begin{tabular}{| c | c | c |}
\hline
\textbf{Matériel} & \textbf{Courant} & \textbf{Puissance} \\ \hline \hline
$4\times$HX711 + Cellules de charge & 26mA &130 mW \\\hline
Total & 113mA & 566mW\\\hline
\end{tabular}
\end{center}
\caption{Consommation réelle (mesurée) à 5V}
\label{tab:consommation}
\end{table}
\end{frame}

\subsection{Prix de revient}
\begin{frame}
\frametitle{\textcolor{white}{Prix de revient}}
\begin{table}
\begin{center}
\begin{tabular}{| c | c |}
\hline
\textbf{Matériel} & \textbf{Prix} \\\hline\hline
Arduino UNO & 20 \euro \\\hline
Shield Bluetooth & 25 \euro \\\hline
$4 \times$HX711 & 1,80 \euro (par pièce) \\\hline
$4 \times$Cellule de charge & 12 \euro \\\hline
Batterie Lithium USB (2000 mAh) & 10 \euro \\\hline
Cable USB-B$\rightarrow$A & 2 \euro \\\hline
Breadbord (plaque prototype) \& Fils & 9 \euro \\\hline
$8 \times$Résistance (1 k$\Omega$) & 0,04 \euro (par pièce) \\\hline
\textbf{Total} &  \textbf{85,52 \euro} \\\hline
\end{tabular}
\end{center}
\caption{Le prix de chaque partie du montage électronique sur \textit{Amazon}}
\label{tab:prix_amazon}
\end{table}
\end{frame}

\subsection{Avantages et évolution}
\begin{frame}
\frametitle{\textcolor{white}{Avantages}}

\begin{itemize}
\item Compatibilité avec plusieurs types de chaises (par exemple, une chaise à 3 pieds)
\item Facile à utiliser
\item Petite consommation électrique (17h d'autonomie avec une batterie de 2000mAh)
\item Compatibilité avec plusieurs plateformes (PC/mobile, Windows/OSX/Linux/Android)
\end{itemize}


\end{frame}

\begin{frame}
\frametitle{\textcolor{white}{\'Evolution}}
\begin{itemize}
\item Miniaturisation: Arduino Uno $\rightarrow$ Arduino Pro Mini;
\item Incruster les capteurs directement dans les pieds de la chaise;
\item L'amélioration de l'autonomie;
\item LEDs RGB et vibreurs au niveau du plateau de l'assise pour notifier l'utilisateur en cas de mauvaise assise;
\item Mode graduel d'avertissement (LEDs, Vibreur et Son);
\item Paramétrage dans l'application mobile.
\end{itemize}
\end{frame}

\subsection*{Bibliographie}
\begin{frame}
\frametitle{\textcolor{white}{Bibliographie}}
\begin{thebibliography}{99}
{\tiny \bibitem{site-arduino} Arduino, \textit{Le site officiel d'Arduino}. 2017. \texttt{\tiny URL : https://www.arduino.cc/} (visité le 26/04/2017).

\bibitem{ntu-swing} Nanyang Technological University, \textit{Java Programming Tutorial - Custom Graphics}. 2016. \texttt{\tiny URL : http://www.ntu.edu.sg/home/ehchua/programming/java/J4b\_CustomGraphics.html} (visité le 26/04/2017).

\bibitem{load-cell} Electronics Stack Exchange, \textit{How to mount a Half Bridge Load cell}. 2016. \texttt{\tiny URL : https://electronics.stackexchange.com/questions/233574/how-to-mount-a-half-bridge-load-cell} (visité le 26/04/2017).

\bibitem{hx711} bodge, \textit{HX711 Arduino library}. 2016. \texttt{\tiny URL : https://github.com/bogde/HX711} (visité le 26/04/2017).

\bibitem{android-studio} Google, \textit{Android Studio}. 2017. \texttt{\tiny URL : https://developer.android.com/studio/index.html} (visité le 26/04/2017).

\bibitem{repo-github} KOKKONIS Dimitrios, LOCHE Jérémy, \textit{pression}. 2017. \texttt{\tiny URL : https://github.com/kokkonisd/pression} (visité le 26/04/2017).

\bibitem{sparkfun} Sparkfun, \textit{Getting Started with Load Cells}. 2016. \texttt{\tiny URL : https://learn.sparkfun.com/tutorials/getting-started-with-load-cells} (visité le 26/04/2017).

\bibitem{wheatstone} National Instruments, \textit{Comment effectuer une mesure avec une cellule de charge ou un transducteur de pression}. 2013. \texttt{\tiny URL : http://www.ni.com/tutorial/7138/fr/} (visité le 26/04/2017).}
\end{thebibliography}
\end{frame}

\begin{frame}
\frametitle{\textcolor{white}{Remerciements}}

\begin{center}
\textit{\huge{Merci pour votre attention !}}
\end{center}

Cette présentation vous a été proposée par :
\begin{itemize}
\item LOCHE Jérémy: \\  \url{jeremy.loche@etu.univ-tours.fr}
\item KOKKONIS Dimitrios: \url{dimitrios.kokkonis@etu.univ-tours.fr}
\end{itemize}

\end{frame}


\end{document}
